\chapter{Temporal Side Channels Part 2} \label{cha:Temporal Side Channels Part 2}

\section*{recap}\label{sec:recap}

In the previous chapter we discussed about RSA cryptosystem~\cite{wikiRSA}. In this chapter, we
presented the most expensive operation in the RSA algorithm which is when we need to take the
message (m) and raise it to the power of the key.
\[ C = m^e \pmod{n}) \] The simplest algorithm for raising a number to a power uses
modular multiplication which is an expensive operation.

\begin{figure}[!ht]
    \centering
    \includegraphics{images/modmul.png}
    \caption{Modular multiplication} \label{fig:modmul}
\end{figure}

Modular multiplication is pretty straightforward. It works just like modular
addition. You just multiply the two numbers and then calculate the standard
name. Examples for Modulo 7 and 15 can be found in \Cref{fig:modmul}. Why
modular multiplication is so expensive? Because we have to take the modulo many
times which is as expensive as division\footnote{Division is the most expensive integer
operation exists, and  therefore we don't want to do many divisions}. So, one way
to do this is by keep multiplying and doing the modular reduction only at
the end. The problem with that approach is that the runtime of multiplication grows exponentially
with the number of the multiplicands. 
Another way is to do a \( mod n \) with each multiplication, but the numbers keep growing 
and the runtime keep raising, so we can't do that either. 

\begin{figure}[!ht]
    \centering
    \includegraphics[scale=0.25]{images/vispow10.png}
    \caption{Visualization of powers of 10 from one to 1 billion.} \label{vispow10:fig}
\end{figure}

As we discussed in the last chapter, the slowest and most trivial way of
implementing modular multiplication \(g^e mod n\) is to take $g$ and multiply it by itself
$e$ times, and every once in a while do modulo $n$, 
(either every multiplication or just in the end). The problem with this is
that if $e$ is a number that consists of 1024 bits, then in the worst case 
 we might need to multiply $g$ in itself \( 2^{1024} \) which is a huge number.

\section{Efficiently Implementing Modular Exponentiations}\label{sec:Efficiently Implementing Modular Exponentiations}

There are two ways for efficiently implementing modular exponentiations:
\begin{enumerate} 
\item performing fewer modular multiplications (instead of  \( 2^{1000} \)  we would
    like to do 1000). 
\item Make each modular multiplication to be less expensive. 
\end{enumerate} 
The best case would be same as regular multiplication over integer.
These two ways will reduce the cost of modular multiplication and modular
exponentiations, and this is something we really want to do in order to make RSA work on
a device.

So, assuming we are engineers, we want to implement modular exponentiations very
cheaply. The right thing to do is obviously use some crypto library, but
let's assume that's not possible, and we are inventing a new CPU. There is a very 
famous book online called the handbook of applied cryptography
(\Cref{fig:appliedCrypt})~\cite{katz1996handbook}. The book is like a recipe book,
and is filled with algorithms, proofs and equations you need
for cryptography. Chapter 14 deals with efficient algorithms for multiplicative
proofs. The book contains several ways to do modular exponentiation. 

\begin{figure}[!ht]
    \centering
    \includegraphics{images/appliedCrypt.jpg}
    \caption{Cover of the handbook of applied cryptography} \label{fig:appliedCrypt}
\end{figure}

The following approach is called the left to right binary exponentiation, also
known as square multiply.
\begin{figure}[!ht]
    \centering
    \includegraphics[scale=0.4]{images/ltrbe.png}
    \caption{The pseudo-code was taken from the handbook of applied cryptography, page 615.} \label{fig:ltrbe}
\end{figure}

The inputs are $g$ (we want to raise g to the power of $e$, \(g^e\)) and \(e\)
which is a bit string of $t$ ($t$ bits), (see \Cref{fig:ltrbe}). 
The most significant bit is always $1$, because there is no
logic behind raising something to the power of zero. In the end of this algorithm \(A\)
will be the result of $g$ raised to the power of $e$. 

In the beginning we set $A$ to be equal to $1$, and go over the bits from left
to right (from the most significant to the least). Each time we square $A$ ($A=A*A$),
but we only multiply by $g$ $(A=A*g)$ if the current bit is $1$ (the \( i^{th}\) bit).
Now, what happens when we do a squaring
operation downstairs? What happens to the exponent upstairs? Its multiplied by
$2$. So \({g^{e}}^2 = g^{2e} \)  and  \(g*g^{e} = g^{e+1} \). Now, we can think
of a binary string, you can write down the bits using shifting (multiplying by
$2$ is shifting) and adding $1$ is just putting one in the place.

So how many modular multiplications will be performed in order to raise g to the
power of $e$? The answer is \(O(t)\). In the best case - what is the lowest amount of 
multiplications that will be performed? the answer
is $t+1$ as the first bit is $1$ (most significant) so we do step 2.2 one time and
all the rest of the bits are zeros, and so we will only do step 2.1 in these cases. 
In the worst cast - what is the highest amount of multiplications that will be performed? 
Answer: $2t$, if all the bits equal to $1$ we will do $2$ multiplications for 
each bit (step 2.1 and step 2.2). In any case, this is equal to \(O(t)\),
so instead of \(2^t\) as in the old algorithm we perform at most $2t$
multiplications. 

Lets review an example. Lets calculate \(7^6=7^{(110)}\) in the group \(
\mathbb{Z}_{15}={1, 2, 4 ,7, 8, 11, 13, 14} \).

\begin{enumerate}
	\item  \(A = 1 = 7^(0) \)
	\item  \(A = A*A = 1 = 7^{(0<<1)} = 7^{(0)}\)
	\item  \(A = A*7 = 7 = 7^{(0+1)} = 7^{(1)}\)
	\item  \(A = A*A = 4 = 7^{(1<<1)} = 7^{(10)} \)
	\item  \(A = A*7 = 13 = 7^{(10+1)} = 7^{(11)} \)
	\item  \( A = A*A = 4 = 7^{(11<<1)} = 7^{(110)} \)
\end{enumerate}

Are there any ways doing this even faster? The answer is yes as we can see in the
handbook of applied cryptography. The general idea of
these algorithms is that we do some pre-computation. The idea of RSA algorithm is
that there is a public key $g$ that is a known number.
So if we know what $g$ is we can prepare all sort of lookup
tables. One method for doing that is called the window method,
where we take three bits at a time, and
instead of doing \(A*g\) we do \( A*g*g\) or \(A*g*g*g\) .. (8 values we can
use), and instead of \( A=A*A\) we do \(A=A*A*A\) and so on, this is the sliding
window. Another method is called the binary method, which is well described in
the handbook of applied cryptography (page 616 algorithm 14.83). We can assume 
that any reasonable crypto implementation is not doing the naive method.
It will use the left to right or the right to left binary exponentiation
which is basically the same idea as one of the window methods.

But what if we want the modular multiplication to be cheaper? A way to achieve this 
is called the Chinese remainder theorem (CRT)~\cite{dingyi1996chinese}, named after 
Sun-tzu Suan-ching that was a teacher from the \nth{5} century and
wrote a book that contained all sorts of riddles and questions.

The Chinese remainder theorem addresses the following type of problem. 
One is asked to find a number that leaves a remainder of 0 when divided by 5, 
remainder 6 when divided by 7, and remainder 10 when divided by 12. 
The simplest solution is 370. Note that this solution is not unique, 
since any multiple of 5 × 7 × 12 (= 420) can be added to it and the result
 will still solve the problem.
The theorem can be expressed in modern general terms using congruence notation.
 Let $n1, n2, …, nk$ be integers that are greater than one and pairwise 
 relatively prime (that is, the only common factor between any two of them is 1),
and let $a1, a2, …, ak$ be any integers. 
Then there exists an integer solution a such that 
$a ≡ ai (mod ni)$ for each $i = 1, 2, …, k$. 
Furthermore, for any other integer $b$ that satisfies all the congruences, 
$b ≡ a (mod N)$ where $N = n1n2⋯nk$. 
The theorem also gives a formula for finding a solution. 
Note that in the example above, 5, 7, and 12 ($n1$, $n2$, and $n3$ 
in congruence notation) are relatively prime. 
There is not necessarily any solution to such a system of 
equations when the moduli are not pairwise relatively prime.

Why does this help us? Allegedly, we have to do 2 operations instead of one. 
The way to do exponentiations in CRT is that we take our big number 
and do modulo $p$ and then modulo $q$ and then there is a CRT step. 
So what is the size of the operand used by these two modular exponentiations?
Assuming that $p$ and $q$ are of the same size? 
Let's say $p$ and $q$ are 1000 bits so the size of $n$ is 2000 bits
(we add the number of bits of each number in the multiplication).  So each
number in the multiplicative group is about 2000 bits because it's mod $N$, so
multiplying two numbers is going to be multiplying two numbers which are 2000
bits. If we reduce it to modulo $p$ and modulo $q$ the numbers are going to be
half the size (1000 bits). So if we take a multiplication and now we will
multiply two numbers that are half the bit length what will be the speed
 improvement? 
 It's times 4. Instead of one big modular exponentiation we have one small
modular exponentiation which costs quarter of the time and another one which
cost quarter of the time followed by a CRT step which is a modular
multiplication of the two. 
How much we spent in total? Answer: a bit more than
half the time, twice speedup. 
Why can't we do that further? Divide $p$ and do it
again? Answer: because $p$ and $q$ are prime numbers and that's the whole point.

On top of that, there is a very nice trick, in order to make modular 
multiplications very cheap and this actually makes modular multiplications 
to be as cheap as regular multiplication.
Multiply two numbers is pretty easy but the
problem is reducing after multiplying. So what if there is a way of doing
modular multiplications without the reduction step? So in 1985 a genius
mathematician called peter Montgomery published a paper called ``modular
multiplication without a reduction step"
~\cite{warren2013hacker}\footnote{https://www.hackersdelight.org/MontgomeryMultiplication.pdf}.
The idea behind the paper is that we enter into a magical world 
called the Montgomery representation. When you step into the Montgomery world, 
modular multiplications do not require a reducing step and when you finish
you just step out of this world and you are back with your result. 
It's still cost you like a multiplication but it doesn't cost you the 
extra reduction step. So, what is the idea of the Montgomery reduction? 
We want to calculate \(g^e \pmod{n}\) and to do that we need to pay a lot 
for modular reduction.
So, the first thing we do is enter the Montgomery representation and do the
Mont(\(g^e\)) and each one of this multiplication steps is going to be about as
difficult as regular multiplication (Figure Figure \ref{montg:fig}). After we
finished with that we exit the Montgomery representation and then we have our
result. Entering and leaving the Montgomery representation costs as much as
modular multiplication but in the middle it's as cheap as regular
representation.

\begin{figure}[!ht]
    \centering
    \includegraphics[scale=0.7]{images/montg.PNG}
    \caption{General sketch of Montgomery Exponentiation} \label{montg:fig}
\end{figure}

Now lets review how the Montgomery Exponentiation works inside
\begin{enumerate}
	\item  Choose a value R, \(R>n\), which is easy to use (usually a large
	power of 2)
	\item  \(Mont(a)=a*R(mod n) =^{def}  a(mod n)\)
	\item  \(Mont(ab)=a*b*R(mod n)=\underline{a}*\underline{b}*R^{-1} (mod n)\)
	\item  \( = (\underline{a}*\underline{b} +
	(\underline{a}*\underline{b}*n'(mod R)*n))/R \pmod{n} \) \newline
	// if this is more than n, subtract n
	\item  \( A = A*A = 4 = 7^{(11<<1)} = 7^{(110)} \)
	\item  Result: Instead of modular reduction, we only (sometimes) subtract
\end{enumerate}

The first thing to do is to choose a very large value $R$ which is larger than
$n$ and should be easy to use ($a$ large power of 2) for instance 1 and 1000
bits of zero. What does it mean easy to use? To multiply and divide by a power
of 2 you just shift left and right. To calculate the modulo of a very large number
that is a power of 2 you do bitwise and with this large power of 2. if
$R=100000$ and $x = 10101010101110$ so to do a modular reduction we just take the
lower bits which are 01110 in this case. So we see it's very cheap operation with
$R$, but $R$ is not useful outside the situation. So to enter the Montgomery
representation we are going to multiply by $R$. This multiplication is modulo
$n$, and this is a bit expensive. But how do we know that $R$ is
inside the multiplicative group? How do we know that multiplying by $R$ doesn't
throw me outside of $mod n$? The answer is: how do we know that 2 is inside the
group? Because what is this group? This group is a multiple of two prime
numbers, and they are odd. Thus 2 doesn't divide either one of them so 2 is in
the group and \(2*2*2...2\) is in the group. We call the new number
\underline{a}. The idea is that the numbers in the Montgomery representation is
cheap so how is it cheap with \underline{a} If we multiply a and b in the
Montgomery's representation It will be: \(a*b*R mod n\). but if I want to do
that using \textbf{a} and \underline{b} we get:
\(\underline{a}*\underline{b}R^{-1}\)mod $n$ because of the extra R. So every
time we multiply two numbers we need to take out the extra R. Inverting $R$ is
simple using GCD and can be done before we start the computation. There are much
more derivations made to get to  \((\underline{a}*\underline{b} +
(\underline{a}*\underline{b}*n'(mod R)*n))/R \pmod{n} \).
\underline{a}*\underline{b} is just a single multiplication.
\underline{a}*\underline{b}*n'(mod R) - Notice that \(modR\) is a cheap
operation because $R$ is 1 with lot of zeroes. ($n$ prime ($n'$) is just precomputed
number that doesn't really matter to us). 
Then we multiply it by $n$ (another multiplication) and then we divide it by $R$ 
which is also a simple operation. So,
we have 4 multiplications and we need a modular reduction which is the main
problem. Montgomery proofed that this sum is no more than \(2^n\). So how do you
do a reduction if the number is between 0 to \(2^n\)? You put an if statement.
If $x$ is less than $n$ you do nothing, if $x$ is larger than $n$ you subtract it
from the number n. So now, instead of division we do a couple of multiplications
over integers which is not that expensive and then we sometimes subtract. This
is a lot cheaper than the basic option and it is widely used. Notice that the if
statement in the algorithm, which is influencing the execution time and might 
enable a timing attack.

\section{Temporal Side Channel on RSA}\label{sec:Temporal Side Channel on RSA}

Let's review how to do a temporal side channel attack on RSA. Kocher described this
attack on his paper in 1995 among other things. If the RSA runs slower than
there are more subtractions and by timing the execution we can recover the key.
So how this can be done? How to recover the key by timing the execution? 
There was a fantastic paper ``A Practical Implementation of the Timing
Attack"~\cite{dhem1998practical} which the remainder of this section is based
on. So what is the game here? Assume you are an attacker who wants to commit a
timing attack on the secure implementation of RSA. If we have some device 
(e.g. a smart card) that we can send him requests, 
for instance ``please allow to pay 1000 dollars
to Alon" (see \Cref{fig:smart card}). The smart card check that I have 1000
dollars in the bank account and then it replies ``I approve this transaction and
sign it". So this card has stored a value inside which means he has money inside.
On the other hand, we want that if we request ``pay Alon 1 million dollars" 
the smart card will say ``I don't have enough funds! I am going to refuse". 
As an attacker we would like to be able to sign any message we want,
mainly messages like ``please send Yossi 1 billion dollars". 
The smart card will not allow it but if we got the private key (signing key)
we can sign whatever message we want. So we send the
smart card a message that is unsigned and he in return sends back a signature
which is the response signed with the private key: \(m^s \pmod{n}\) where $s$ is the
secret key. We assume that as an attacker we can send as many queries as we want and we
and recover the responses (the signatures). The goal is to extract the
secret key. 

\begin{figure}[!ht]
    \centering
    \includegraphics[scale=0.4]{images/smartcard.png}
    \caption{The timing attack principle} \label{fig:smart card}
\end{figure}

So, let's look a bit closer on the attack model - what messages we can send to
the smart card? Can we send any message we want? The answer is that we can only send
valid messages - if the message is not valid then the smart card will just throw 
an exception.
This is somewhere in the scale between the weakest model and the most
powerful model. In this scenario the most permissive attack model is a known plain
text. Known plain text means that we can see the messages as they are go in to the
smart card but cannot change them. So the next thing is choosing the plain text. 
We can't just chose any plain text, it has to follow a certain rule. The next
thing is completely chosen plain text which doesn't enforce any rules, and the
last thing is adaptive plain text which means we can look at the response and we
can choose the next query that we will send. So, what is the attack model, we send
requests, the smart card is signing them, and we get the responses and also
assuming that the smart card is using Montgomery RSA . So how can we use it to
extract the key? We are going to use a method
called ``Vaizata" method which can be found in the DPA handbook. This is a
general way of performing a side channel attack using statistics. \newline

\underline{The ``Vaizata" Method}
\begin{itemize}
	\item Make a simple assumption about the implementation
	\item Guess a little part of the key
	\item Make hypothesis about the effect of the guess on the execution
	\item Classify the measurements according to the hypothesis
	\item If we guessed right, the classification will be statistically
	meaningful
\end{itemize}

How does it work? First, we make a simple assumption on the implementation, an
assumption could be: We assume that the implementation run on software (the
other possibility is hardware) what does it gives us? Software is executed
serially and in the hardware it's not the case. We can assume for example that
the key is stored in a flash memory on the device, and every time we need to use
the key then we need to read the flash memory. How can we find that this is the
case? How can we find first of all that a device is on using the hardware or the
software? One way is to look at it - we can open the screws and look with a
microscope, or we can go to this wonderful website called ``I fixed it". They
disassemble all sorts of devices and share this information. What is the sign
that the device is using software? If there is an update on the firmware,
because when it wakes up it needs to find out what software to run. We can say
that this device uses memory, and we can also say things about when the 
device is doing the encryption. 
Let's assume the device under test is a remote controller.
Inside the controller there is a secret key stored and also a counter. Whenever
the button is pressed the remote controller constructs a package containing the
serial number of the controller, the counter and a boolean state of the
button (which ever button was pressed). Then, it sends it to a car and the car
decrypts it. Why do we need a serial number? Each ECU in the car has a 
program to accept remote controls. So why do we need the counter? Without the
counter, an attacker can repeat a message sent from the remote controller to the
car only by reading the messages and sending them again. What happens when you
press the button and the car is in the train station? The counter in the remote
controller is out of sync with the counter in the car ans there is a window of
counter values the car will accept. If its closed, the car will open without
complaining. If it's a little closed we will need to press the remote control
twice and when the car will see consecutive values and it will open,
 but if it is too far it won't open. 
 Let's assume we can find out when the controller is
transmitting (it is a very intensive operation that takes battery life and also
radiates). So we know the moment in time when it is transmitting, did the
encryption happened before or after the transmission? The answer is before. Let's
assume that the counter stored in the memory and let's say we found the moment in
time when the chip is reading from memory. Did it happen before or after the
encryption? The answer is after, since we need the counter to be included in the
message that will be sent. We can also make more assumptions such as ``the AES
uses an 8 bit data pack or 16 bit or 32 bit". Some of these assumptions might be
wrong but the ``Vaizata" method will help us to find out if they are wrong. 

So first of all we make a simple assumption, that the smart card is using
left to right binary exponentiation using Montgomery and this is a very
reasonable assumption because most implementations are using this method.
The next thing to do is recursively (or inductively) guess a little part of the key.
if we guessed the whole password it would take us exponential time but if 
we could guess a small part of the key each time it would take a linear time. 
So, we will try to guess small parts of the key first, and maybe the simplest thing 
to explain is guessing one bit or a single character, but lets say we are guessing a small part
of the key. So now we know the beginning of the key and there is a little part
we don't know. The next step is that we need to make a guess about what kind of 
effect our guess is going to have on the computation. We can say, for example,
that if we will guess the bit correctly then something will happen to the
computation, and if we will guess this key bit correctly it will take more
power/take longer time/connect to the network more often or some kind of other 
phenomena we can measure. So, in our case what is the only thing we can measure? Answer:
The answer is time. We assume that if there is a Montgomery reduction in the calculation of
this bit then the entire computation is going to take a little more time.


SO what is going on here? There is a very large computation here and we were able
to guess the beginning of it but not the end of it. Now we are going to classify
the measurements according to our hypothesis, in this case two groups (with or
without Montgomery computation). If we guessed correctly then it will take
longer to the group we said it will take longer. If not, it might take less or
more, we can't determine. If we guessed correctly, the groups will have meaning,
means will be able to statistically tell apart the set of the measurements
that will take a longer time and the set of measurements that will take less
time. We are going to guess the left bit of the key, and there is only one option.
The next bit can be 1 or 0. Now, we can
simulate the running of this algorithm with our guess, not for the whole key but only
to the part we know, and if there is going to be Montgomery reduction. So assuming we
have $g$, $e$ and $N$. The device under test is calculating \(g^e \pmod{n}\), but where $g$ came
from? It is supplied by the attacker. What about $e$? Secret we want to discover
(\(e_t,e_{t-1},..,e_0\)) What about $N$? public variable (known). The first
thing it does, it enters the Montgomery representation. This information of how
to enter the Montgomery representation is known to the attacker. It starts with
$g$, and then $g$ becomes $Mont(g)$, but the attacker can also do it. First \(A=1\), then
\(A=A*A\) the next thing is \(A=A*g\) because we know that the most significant
bit is 1. The next step is \(A=A*A\), now what is next? It depends, if
\(e_{t-1}\) = 0 then \(A=A*A\) (skipping to next bit). Else if \(e_{t-1}\) = 1
then \(A=A*g\). What happens next now we can't know. We can run both
calculations, in particular we can find out if there was a reduction step in two
optional operations. There are 4 options: Only one of then containing a reduction step,
two of them containing a reduction step or neither of them. We can know exactly, 
assuming we make a guess on \(e_{t-1}\),
if there is going to be an extra reduction step. We know enough to guess - if we
guess correctly, we can know if there will be a reduction step because we can
calculate all the alternative options completely. So, let's do this now, we have
many different g's, and we have repeated this step many times, and for each of these
g's we know if there is going to be an extra reduction step. So, now let's see
how we can do an attack using this information. So, there is private key $s$, 
a public key $v$ and a signing operation \(m^s mod n\). 

We begin the attack with a bag of messages, all of them are valid, and we send
them to the device under test (DUT). The DUT signs these messages (k messages),
and for each one of the messages we get a trace, which is the data we collected
using the side channel attack in this case it is only the time. Now we have a
vector of size k and each element in the vector is the time it took to sign the
message. Now, we are going to try and guess \(s_t,s_{t-1}\) \((s =
s_t,s_{t-1},s_{t-2},…,s_0)\) and try to discover \(s_{t-2}\). 

So for each of the messages and each key guess we are going to simulate the
computation as far as we already know and in addition for the parts of the key
that we don't know we are going to simulate twice - one with a 0 and one with a
1. We are going to find out where the extra reduction step happens. If the next
bit is 0, then some of the messages have extra reduction and we classify the message into
two bins, those who got extra reduction and those who didn't. But maybe the key
is not 0? maybe its 1. So we can simulate the same thing with the next bit as 1,
and find out different set of messages with an extra reduction step. We can't
find what the bits are but we can make a guess and simulate on both 0 and 1. So,
now we divide our traces into two groups in 2 different ways, if the next key
bit is 0 then \(m_1,m_2,m_4,m_5,m_7\) got extra reduction and \(m_3,m_6,m_8,m_9\)
didn't. If the next key bit is 1 then \(m_2,m_3,m_5,m_8\) got extra reduction and
\(m_1,m_4,m_6,m_7,m_9\) didn't (see \Cref{fig:extraRed}). 


\begin{figure}[!ht]
    \centering
    \includegraphics[scale=0.3]{images/extraRed.PNG}
    \caption{Key bit guess simulation} \label{fig:extraRed}
\end{figure}

If we guessed correctly what can we tell about the messages that got an extra
reduction? Their runtime will be a little longer if we guessed the key bit
correctly and If we guessed incorrectly it means we divided into two random
groups which means the runtime will be similar in both groups. So if we guessed
correctly the difference in runtime between the groups will be measurably
different. So now we are going to change the discussion about the messages into
the traces.

\begin{figure}[!ht]
    \centering
    \includegraphics[scale=0.3]{images/extraStat.PNG}
    \caption{Guessing correctly the key bit makes the statics measurably different} \label{fig:extraStat}
\end{figure}


if the next key bit is 0 then the statistics of the runtimes of 0 bit with extra
reduction are going to be different from the runtimes of 0 bit without extra
reduction. If we were wrong then the runtimes of 1 bit with extra reduction will
be different from 1 bit without the extra reduction. Notice that we are not saying
average or mean anywhere, because they are not required, it could be the
variance changes or something else. The point is that there is some kind of
difference that we can measure. So, how can we find out which of these two
divisions is the correct one? Answer: we have two divisions of k traces and we
measure the distance of the means. We are going to calculate the mean runtime of each
part (0 bit with or without the extra reduction and 1 bit with or without the
extra reduction) and subtract between with/without extra reduction in each bit
guess. If there is a large distance of means of the 0 bit guess as oppose to the
1 bit guess then probably the splitting of traces in the 0 bit guess is more
meaningful than the 1 bit guess. If we are able to split meaningfully then we
guessed the key correctly. Now, let's go back into statistics and talk about the
T-test~\cite{wikittest}. The T-test was invented by William Sealy
Gosset~\cite{wikigosset} who was a chemist who worked for a very famous brewery
in Ireland (Guinness). So, what is the idea? We have two populations that are
different in some way. There are two kinds of t-tests, pair T-test and unpaired
T-test. what is the paired T-test? lets assume we are going to a
tree and we taking the leaves that fall off the tree. We notice that the leaves
that fall on the south side have less mold that the leaves that fall
on the north side. Why is that? Because there is more sun on the south side that
is drying the leaves. 

How do we prove our theory? We send our undergrad research assistant to collect
a bag of leaves form the north side and a bag of leaves from the south side and 
tell the undergrad to count the percentage of mold in the southern and northern
leaves. When the undergrad comes back, very exhausted, he provides us with an
excel file that have 1000 leaves from the north side and 1500 from the south
side. Now we want to prove our theory, each one of the leaves has a mold, We
want to prove that there is a statistical difference between the two groups - 
this is the unpaired T-test. What is the paired T-test? 
It is when we are talking about
something which we can identify as pairs. For example, we are developing a
cancer medicine and we want to test it. Now, we don't take any undergrads but
only very sick mice with cancer. We measure the weight of the tumor in order to
have a list of 100 mice and tumors. Then we divide them into two groups, one we
treat with a medicine and the other we don't. In the end of the experiment, we
measure the tumors again, but now each measurement is a pair, one before the
treatment and one is after. We want to say that the size of the tumor is
smaller after the treatment. Now let's review the student's unpaired T-test demo
in Matlab. The mat in Matlab is for matrix, we can define matrix like this:

\begin{figure}[!ht]
    \centering
    \fbox{
    \includegraphics[scale=0.5]{images/defmat.png}}
    \caption{Defining a matrix of size 1x10 with increasing numbers from 1 [1,2,3,…,10]} \label{fig:defmat}
\end{figure}

\begin{figure}[!ht]
    \centering
    \fbox{
    \includegraphics[scale=0.5]{images/defmat2.png}}
    \caption{Adding the matrix with 2 increases all the numbers in the matrix by 2 (Same with multiplication and log)} \label{fig:defmat2}
\end{figure}

\begin{figure}[!ht]
    \centering
    \fbox{
    \includegraphics[scale=0.5]{images/defmat3.png}}
    \caption{\(x>5\) will result with logical array [0, 0, 0, 0…, 1, 1, 1, 1]} \label{fig:defmat3}
\end{figure}

The function randn(1) which creates normally distributed random variable which
follows a Gaussian distribution which means the variance is 1 and the mean of 0.
What is the minimum value this function will output? Answer: None, but
statistically the value is closer to 0. How can we create a random variable with
mean different from 0? Answer: add a constant to the mean, if we want mean N. we will do $N$ +
randn (1). We can even create a vector of randomly chosen numbers using randn
(1, 5) + 5 : [4.9369, 5.7147, 4.7950, 4.8759, 6.4897] and if we will do randn
(1, 5)*.1 + 5 the numbers will be even closer to 5. So, now we are ready to try
the student T-test.

\begin{figure}[!ht]
    \centering
    \fbox{
    \includegraphics[scale=0.6]{images/defmat4.png}}
    \caption{In the code, 2 vectors are created - x and y in the same size (which is not obligatory), $x$ will be randomly distributed with a mean of \(mu_{1}\) and variance of \(sigma2_1\) and vector $x$ is going to be random distributed with a mean of \(mu_2\) and variance of \(sigma2_2\). Then there is a code for plotting both vectors (in different colors).} \label{fig:defmat4}
\end{figure}

\begin{figure}[!ht]
    \centering
    \includegraphics[scale=0.6]{images/defmatplot.png}
    \caption{Plotting of the vectors for \(mu_1 = 10, mu_2 = 20\)} \label{fig:defmatplot}
\end{figure}

Looking only on the graphs - can we say they are from the same distribution?
Answer: we can't be sure. We want to run the T-test to find out if they are
statistically significant. There is a hypothesis and we need to either reject or
accept the hypothesis. If \(H_0\) then they are from the same distribution and
if \(H_1\) than they are from different distributions. If we run the code, the
result will be that they are from different distributions with 0.959 certainty.
Each time we run we can get different results, if the certainty is smaller than
0.95, the hypothesis is rejected. How can we make it more difficult
for the ttest2? Answer: if we change \(mu_2\) to 11, then they will be very
close distributions. 

\begin{figure}[!ht]
    \centering
    \includegraphics[scale=0.6]{images/defmatplot2.png}
    \caption{Plotting of the vectors for \(mu_1 = 10, mu_2 = 11\)} \label{fig:defmatplot2}
\end{figure}

How we as attackers can handle this situation? Answer: run many times. So we
change the vector length (for example to 50000). There is something called power
analysis, which means given these parameters ($mu$ and $sigma$) how many
measurements you need to run to be sure with 95 percent. As engineers we don't really
care about T-tests, we have a budget of measurements and we just take the one
with the larger mean distance, but what if we are wrong (guessed 1 instead of
0)? After we guessed one bit wrong all the bits afterwards are wrong
because they are simulated wrongly. So, how can we simulate a full attack? If we
guessed 1 bit using the differences of means then we continue to the next bit in
linear time. Let's review a figure from " A Practical Implementation of the
Timing Attack" which you are encouraged to read.


\begin{figure}[!ht]
    \centering
    \includegraphics[scale=0.25]{images/figpita.png}
    \caption{x axis – is the bit index from right to left (the left most bit is known to be 1) and y axis – is the distance of means we chose (some times its 500 or 100 but after 151 the distance of means is a lot smaller which means we guessed a bit wrong). How to solve it? Answer: go backwards and backtrack.} \label{fig:figpita}
\end{figure}

Now let's talk about counter measurements. When Kocher announced the attack to
the cypherpunks mailing list there was kind of discussion about it. Here is a
message (see \Cref{fig:paul}) from there that was sent by Ron Rivest. He was
replying to William Simpson, who was the author of Photuris which is related to
the IPsec protocol and was used for key exchange in the IP protocol. 

\begin{figure}[!ht]
    \centering
    \includegraphics[scale=0.25]{images/paul.png}
    \caption{The message.} \label{fig:paul}
\end{figure}

When he read that this attack can attack Photuris, Bill replied: don't worry
this will be fixed in Photuris. How to do this? By dithering the return time of
identification message a few extra milliseconds. Which means he is doing
mitigation, as he is not adding a random delay because random is very expensive, he
is going to finish the calculations and is going to look at the clock and
exactly when the millisecond changes (or second) send the package. What does it
mean? It means that since the beginning is random then he is going to add a random
delay. So, what Ron Rivest replied? It will reduce the data leakage but will not
eliminate it. Why so? How the attacker will overcome this? He will need
to measure more, this is network-based protocol, the attacker can measure as 
many times as he wants. He says, in addition, the public key computation time should be
constant and independent from the message being sent. Ron Rivest suggest
prevention as a countermeasure. So, what time it is going to be? Answer: the
worst-case time. he adds a side note that this kind of attack is very difficult
to mount in an internet environment due to packet-routing timing variabilities,
however it is wise to be careful. Few years later a demonstration was presented
over the internet, there were more measurements to counter the routing problem.
Adding noise is sometimes the only thing that works, if you add enough noise to
delay the attack time up to a year the message might not be relevant by the time the attack succeeds.

So, let's talk about two more counter measurements, which are preventions. The first
one is called RSA blinding. RSA blinding is a prevention counter measure which
actually works on average time and not on worst-case time. if we have a secret key $S$ and
want to calculate \(m^s mod n\), and the attacker gives us $m$ and we don't want
to leak $s$. We showed that with enough attempts from the attacker he will eventually
retrieve $s$. 

So, how do we do blinding? First of all, we can do this even before the attacker
arrives, we generate random $r$ and calculate \(r^v mod n\) and \(r^{-1} mod
n\). These calculations are not reviling any secrets because $v$ is the public
key. The attacker gives us $m$, so we calculate: 

\[X = (r^v * m) mod n\]
\[Y = X^s = (r^v*m)^s = r^{vs}*m^s = r*m^s mod n \] 
\[ // v*s mod n = 1 mod n)\]

Now, $Y$ is leaking information because we raise a number to the power of the
secret key, but $r$ is a random number which the attacker doesn't know so he can't
simulate the execution. How we remove $r$? We just calculate 

\[S = Y*r^{-1} = r*m^s*r^{-1} = m^s mod n\]

Why doesn't everybody use it? Answer: because it's expensive, 2 modular
exponentiations instead of 1. Another problem is the random number generation,
its hard to find random number generator. You can see RSA blinding in openPGP. 

Now let's review another countermeasure. It called square and always multiply
(see \Cref{fig:saama}). It is very similar to the square and multiply, just
instead of only when \(d_i\) is 1 we will always calculate the multiply but the
assignment will be only when \(d_i\) is 1. The problem with this is not the
extra computation that came from turning sometimes to always. Speculative
execution is always looking for instruction to execute, how does it decide if it
will execute an instruction? If all it's dependencies are met. If I say \(a =
b*c\) and \(e = b*c\) they can run both in the same time. What happens is when
the instruction brought to the CPU there is actually nobody is waiting for $t$, so
as soon as it finishes to run the \(s =s *s mod n\) and \(d_i\) is equal to 0 it
will just return $s$. Moreover, the compilers are also capable of detecting such
cases and optimize them by dismissing the else statement. So if we have a very
simple CPU with no speculative execution no compiler and we wrote it in assembly
we won't be able to attack it, but we could use power analysis.

\begin{figure}[!ht]
    \centering
    \includegraphics[scale=0.15]{images/saama.png}
    \caption{Square and always multiple algorithm} \label{fig:saama}
\end{figure}

If we run this algorithm as is, we have two places in memory for $s$ and for $t$,
every action we load $s$ then multiply and then edit the memory of $s$. Same goes in
the multiply section. We load $s$ and $m$ and then multiply and store it in $s$. It's
always loading and storing $s$, but what happens when it goes to the else statement:
it loads $s$ and $m$ and then store this into something other then $s$. So the power
consumption is different between the store and the load. You can read more in
the paper (``defeating RSA multiply-always and message binding" by Marc F.
Witteman et al.~\cite{witteman2011defeating})

Side channel attacks can get not only computer secrets but human secrets too.
What exactly is a human secret? Browsing history for example. How does the
website figure out our browsing history? Theoretically, we can delete the
history in the browser. But what happens when we click on a hyper link (blue
link)? It turns purple after the click. So, there was a nice trick that websites
used to do, they attached the link to an HTML element and added JavaScript code
that at the change of the color can now update that you have visited the site.
The world wide web consortium decided that it was a privacy leak and you are not
allowed to read the color of an HTML element anymore. Now we can set the color
but can't read it. One of the speakers in the black hat 2013 used timing attacks
to find out if a web site was visited or
not\footnote{\url{https://www.youtube.com/watch?v=KcOQfYlyIqw}}. He also
demonstrated how he can also use timing attack to read the user's stream. 
\section{Research highlights} \label{sec:TemporalSCRelatedWork}
\begin{itemize}
    \item \textbf{Photonic Side Channel Attacks Against RSA} - This paper describes an attack utilizing the
    photonic side channel against a public-key crypto-system. They
    evaluated three common implementations of RSA modular exponentiation, all using the Karatsuba multiplication method.
    It was discovered that the key length had marginal impact on resilience to the attack.
    They noticed that the most dominant parameter impacting the attacker’s effort is the minimal block size at which the Karatsuba method reverts to naive multiplication.
    They also discovered that Montgomery’s Ladder was actually the most susceptible to the attack.
    \href{https://www.eng.tau.ac.il/~yash/ieee-host-2017.pdf}{Photonic Side Channel Attacks Against RSA}.

    \item \textbf{Thermal Covert Channels on Multi-core Platforms} - This paper demonstrates a way to leak information between 2 applications which are running on the same machine but with complete CPU core isolation or timing separation.
The paper is presenting a way to bypass the first tactic by utilizing the heat factor and describing an attack where one CPU core is generating heat because of running a CPU intensive task (such as RSA decryption looped for 100 milliseconds) and the nearby CPU core picking up the heat as it propagates from the original core to its neighbors (by the laws of physics).
The way to bypass the second type of isolation is just like the first one, but in this example the applications are not running at the same time. Still, the second application can track the remanent heat generated from the first application.
The paper then offers a future work where an application can use these methods in combination with machine learning techniques to be able to learn the purpose of other applications running on the same machine without having the permissions to do so.
    \href{https://www.usenix.org/conference/usenixsecurity15/technical-sessions/presentation/masti}{Thermal Covert Channels on Multi-core Platforms}.

    \item  \textbf{Summarize of Drones’ Cryptanalysis -} 
    \\ \textbf{Smashing Cryptography with a Flicker Paper}
    This paper \cite{nassi2019drones} addresses the question how we can tell whether a passing drone is being used by its operator for a legitimate purpose (e.g., delivering pizza) or an illegitimate purpose (e.g., taking a peek at a person showering in his/her own house).\\
    \\Over the years, many methods have been suggested to detect the presence of a drone in a specific location, however since populated areas are no longer off limits for drone flights, the previously suggested methods for detecting a privacy invasion attack are irrelevant.\\ 
    \\In this paper, By applying a periodic physical stimulus on a target/victim being video streamed by a drone,  a new method that can detect whether a specific POI (point of interest) is being video streamed by a drone is presented. 
    Based on this method, an algorithm for detecting a privacy invasion attack is presented.\\ 
    The paper analyse the performance of the algorithm using four commercial drones. \\
    \\It show how the method that been suggested can be used to determine whether a detected FPV (first-person view) channel is being used to video stream a POI by a drone, and locate a spying drone in space.\\
    \\The evaluation on algorithm that presented in the paper shows that a privacy invasion attack can be detected by the system in about 2-3 seconds.
    
    \item  \textbf{Summarize of Whispers in the Hyper-space:}
    \\ \textbf{High-speed Covert Channel Attacks in the Clouds} 
    In the paper \cite{cloud_covert_channel_usenix12} the authors are presenting the usage of two covert channels to communicate between two virtualized x86 systems that run on the same physical machine, they develop a robust communication protocol to deal with high noise in the covert channel and achieve information transfer of more than 100 bps with 0.75% error rate. The researchers had also implemented the attack on Amazon EC2 cloud services.
    The researchers went over the different x86 processors’ generations and showed that they can create a covert channel with each one, whether it contains a memory bus (new) or not (old). The first covert channel type they showed is based on using the bus lock mechanism. According to the implementation of an atomic instruction, the sender can lock the bus when using the instruction. locking the bus has effects over the whole system and the receiver can measure the latency when getting some variables and determine if 1 or 0 was sent by the receiver.
    The second covert channel type is using atomic instruction over two cache lines when no bus lock mechanism exists. The atomicity is achieved by flushing all inflight memory transactions from all cores which will result in observable latency.
    This work had shown how to solve three obstacles when creating a covert channel in the cloud, first memory addressing uncertainty- the sender and the receiver cannot know the exact cache locations of each other because they are located in different virtual environments with different memory mappings. Second process scheduling uncertainty - in classic cache covert channel the receiver runs after the sender (round-robin scheduling), but when the two processes located in different VMS, this scheduling is not guaranteed. Third physical limitation - processes of the receiver and sender may run on different cores and may not share caches L1 and L2.

    \item \textbf{Cross-Origin Pixel Stealing: Timing Attacks Using CSS Filters} - In the paper \cite{kotcher2013cross}, the authors present the
    following threat model: the attacker runs a malicious domain. His purpose is to tempt his victim to use the malicious domain. He either creates
    an interesting website to attract the victim to use it for a while,or he is able to open another window in front of the window being attacked.
    \\\\
    Hypertext Markup Language (HTML) is the standard markup language for documents designed to be displayed in a web browser. The Document Object Model
    (DOM) is a cross-platform and language-independent interface that treats an XML or HTML document as a tree structure wherein each node is an object
    representing a part of the document. It is shown in the paper, how the rendering process of DOM content makes timing attacks possible: CSS filters
    allow styling of HTML components, CSS Custom filters (a.k.a shaders) have access to the rendering content, which can hold sensitive information, so
    an attack on shaders could lead to information leaks. But even without this access, different type of code parts in shaders could enable timing attacks.
    \begin{enumerate}
        \item \textbf{Expand a single pixel} - We need to choose which pixels we want to steal, and in each iteration of the attack we need to take only
        one pixel from the image and enlarge it, to be at the size of the whole screen in order to cancel the noise of all the other pixels and distinguish
        between only one black and white pixel. Worth mentioning that a method has been found to be able to find the exact character out of 16 digits by
        measuring the value of only 4 bytes, and in general character set of size $N$ can be read by testing only $log_2(N)$ pixels.
        \item \textbf{Framing a page} - The attacker frames a website that has neglected to use X-Frame-Options.
        \item \textbf{The victim visits a malicious page} - The victim visits the attacker’s malicious web page and is tricked into remaining on the page
        for the duration of the attack, e.g. by an interesting advertisement.
        \item \textbf{Traversing the page} - Activate a combination of CSS filters over the page to convert each value to be RGB(0,0,0) - black,
        or RGB(255,255,255) - white.
        \item \textbf{Average framerate is captured} - Using requestAnimationFrame \cite{requestAnimationFrame} to determine the average framerate on the
        browser window for each target pixel.
        \item \textbf{Interpreting the captured data} - An array of pixel measurements are sent to the attacker’s server to be interpreted, The attacker can
        even build a classifier to determine the  exact results based on the measurements.
    \end{enumerate}
    As a result of this attack, and serval similar attacks \cite{stone2013pixel} \cite{andrysco2015subnormal} discovered at the same time, all web browsers had changed their filters implementation to be
    without color-related optimization and all calculations are supposed to take the exact same time. Despite that, people are always finding new ways
    to steal pixels through other, creative, ways e.g. floating-point or hardware implementation of the CPU. And so, the fight for security is still ongoing to this day.

\end{itemize}
